% Template for white paper submissions for the 
% LSST Call for Observing Strategies for DeepDrilling and Minisurveys 
% 
% The call for white papers can be found at https://github.com/lsst-pst/survey_strategy/blob/master/latex/WPcall2018.pdf
% The deadline for submissions is November 29, 2018
% Please submit your white paper proposal via a pull request at https://github.com/lsst-pst/survey_strategy_wp, after creating a 
%   subdirectory named LASTNAME_FIRSTNAME_NUMBER
% For help with white papers or the submission process, please post at http://community.lsst.org/c/sci


\documentclass[11pt]{article}
\usepackage[utf8]{inputenc}
\usepackage{booktabs}

\title{Template for LSST Call For Survey Strategy Optimization White Papers}
\author{}
\date{March 2018}

\begin{document}

\maketitle

\begin{abstract}
Please provide a short summary of your proposed observations and scientific goals here.
\end{abstract}

\section{Scientific Motivation}

\begin{footnotesize}
{\it Describe the scientific justification for this proposal in the context
of your field, as well as the importance to the general program of astronomy. [page limit]}
\end{footnotesize}

\vspace{.6in}

\section{Technical Description}
\begin{footnotesize}
{\it Describe your proposed observations. Please comment on each observing constraint
below, including the technical motivation behind any constraints. Where relevant, indicate
if the constraint applies to all requested observations or a specific subset.}
\end{footnotesize}

\subsection{High-level description}
\begin{footnotesize}
{\it Describe or illustrate your ideal sequence of observations.}
\end{footnotesize}

\vspace{.3in}

\subsection{Footprint -- pointings and/or constraints}
\begin{footnotesize}{\it Describe the specific pointings or general region (RA/Dec, Galactic longitude/latitude or 
Ecliptic longitude/latitude) for the observations. Please describe any additional constraints, especially if there
are no specific constraints on the pointings (e.g. stellar density, galactic dust extinction).}
\end{footnotesize}

\subsection{Image quality}
\begin{footnotesize}{\it Constraints on the image quality (seeing).}\end{footnotesize}

\subsection{Sky brightness and/or individual image depth}
\begin{footnotesize}{\it Constraints on the sky brightness in each image and/or individual image depth for point sources.
Please differentiate between motivation for a desired sky brightness or individual image depth (as 
calculated for point sources). Please provide sky brightness or image depth constraints per filter.}
\end{footnotesize}

\subsection{Co-added image depth and/or total number of visits}
\begin{footnotesize}{\it  Constraints on the total co-added depth and/or total number of visits.
Please differentiate between motivation for a desired co-added depth and/or the total number of visits.
Please provide desired co-added depth or number of visits per filter, if relevant.}
\end{footnotesize}

\subsection{Grouping of visits}
\begin{footnotesize}{\it Constraints on the number of exposures in a visit, and the number of visits in a night. 
If these groupings should be obtained in particular sets of filters, please describe.}\end{footnotesize}

\subsection{Timing of visits}
\begin{footnotesize}{\it Constraints on the timing of visits --- within a night, between nights, between seasons or
between years (which could be relevant for rolling cadence choices in the WideFastDeep. 
Please describe optimum visit timing as well as acceptable limits on visit timing, and options in
case of missed visits (due to weather, etc.). If this timing should include particular sequences
of filters, please describe.}\end{footnotesize}

\subsection{Filter choices}
\begin{footnotesize}
{\it Please describe any special filter requests not included above.}
\end{footnotesize}

\subsection{Exposure constraints}
\begin{footnotesize}
{\it Describe any constraints on the minimum or maximum exposure time required (or alternatively, saturation limits).}
\end{footnotesize}

\subsection{Other constraints}
\begin{footnotesize}
{\it Any other constraints.}
\end{footnotesize}

\subsection{Estimated time requirement}
\begin{footnotesize}
{\it Total time requested for these observations, assuming overheads as described at ??.}
\end{footnotesize}


\begin{table}[htb]
    \centering
    \begin{tabular}{l|l|l|l}
        \toprule
        Properties & Rank & Constraint Summary \\
        \midrule
        Image quality &   &    \\
        Sky brightness &  &  \\
        Individual image depth &  &   \\
        Co-added image depth &  &   \\
        Number of exposures in a visit &  &   \\
        Number of visits (in a night) &  &   \\ 
        Total number of visits &  &   \\
        Time between visits (in a night) &  &  \\
        Time between visits (between nights) &  &   \\
        Long-term gaps between visits & & \\
        \bottomrule
    \end{tabular}
    \caption{Summary of observation properties. Rank (specify how we want ranking to work)}
    \label{tab:obs_constraints}
\end{table}

\subsection{Technical trades}
\begin{footnotesize}
{\it To aid in attempts to combine this proposed survey modification with others, please address the following questions:
\begin{enumerate}
    \item What is the effect of a trade-off between your requested survey footprint (area) and requested co-added depth or number of visits?
    \item If not requesting a specific timing of visits, what is the effect of a trade-off between the uniformity of observations and the frequency of observations in time? e.g. a `rolling cadence' increases the frequency of visits during a short time period at the cost of fewer visits the rest of the time, making the overall sampling less uniform.
    \item What is the effect of a trade-off on the exposure time and number of visits (e.g. increasing the individual image depth but decreasing the overall number of visits)?
    \item What is the effect of a trade-off between uniformity in number of visits and co-added depth? Is there any benefit to real-time exposure time optimization to obtain nearly constant single-visit limiting depth?
    \item Are there any other potential trade-offs to consider when attempting to balance this proposal with others which may have similar but slightly different requests?
\end{enumerate}}
\end{footnotesize}

\section{Performance Evaluation}
\begin{footnotesize}
{\it Please describe how to evaluate the performance of a given survey in achieving your desired
science goals, ideally as a heuristic tied directly to the observing strategy (e.g. number of visits obtained
within a window of time with a specified set of filters) with a clear link to the resulting effect on science.
More complex metrics which more directly evaluate science output (e.g. number of eclipsing binaries successfully
identified as a result of a given survey) are also encouraged, preferably as a secondary metric.
If possible, provide threshold values for these metrics at which point your proposed science would be unsuccessful 
and where it reaches an ideal goal, or explain why this is not possible to quantify.}
\end{footnotesize}

\vspace{.6in}

\section{Special Data Processing}
\begin{footnotesize}
{\it Describe any data processing requirements beyond the standard LSST Data Management pipelines and how these will be achieved.}
\end{footnotesize}

\end{document}
